\documentclass[10pt]{article}
\usepackage{amsmath, listings, amsthm, amssymb, latexsym, proof, syntax, stmaryrd, tikz-cd}
\usepackage[T1]{fontenc}
\usepackage{scrlayer-scrpage}

\newtheorem{theorem}{Theorem}
\newtheorem{definition}{Definition}

% \renewcommand{\syntleft}{}
% \renewcommand{\syntright}{}

\newcommand{\infers}{\ensuremath{\Rightarrow}}
\newcommand{\checks}{\ensuremath{\Leftarrow}}

\newcommand{\ttt}[1]{\texttt{#1}}
\newcommand{\ra}{\ensuremath{\rightarrow}}
\newcommand{\Ra}{\ensuremath{\Rightarrow}}

\newcommand{\reduces}{\ensuremath{\longmapsto}}
\newcommand{\reducestr}{\ensuremath{\longmapsto^{*}}}

\newcommand{\BigStep}{\ensuremath{\Downarrow}}

\newcommand{\val}{\textnormal{ val}}

\newcommand{\True}{\ttt{True}}
\newcommand{\False}{\ttt{False}}
\newcommand{\head}{\ttt{head}}
\newcommand{\tail}{\ttt{tail}}
\newcommand{\cons}{\ttt{cons}}
\newcommand{\nil}{\ttt{nil}}
\newcommand{\unit}{\ttt{unit}}
\newcommand{\foldr}{\ttt{foldr}}
% \newcommand{\scanr}{\ttt{scanr}}
\newcommand{\pair}{\ttt{pair}}
\newcommand{\fst}{\ttt{fst}}
\newcommand{\snd}{\ttt{snd}}
\newcommand{\ite}{\ttt{if}}

\newcommand{\Int}{\ttt{Int}}
\newcommand{\Bool}{\ttt{Bool}}
\newcommand{\tyUnit}{\ttt{Unit}}
\newcommand{\tyPair}{\ttt{Pair}}
\newcommand{\List}{\ttt{List}}
\newcommand{\letbnd}{\ttt{let}}
\newcommand{\inexpr}{\ttt{in}}
\newcommand{\defeq}{\ttt{:=}}

\newcommand{\intS}{\ttt{int}}
\newcommand{\intset}{\ttt{intset}}
\newcommand{\bool}{\ttt{bool}}
\newcommand{\pred}{\ttt{pred}}

\newcommand{\true}{\ttt{true}}
\newcommand{\false}{\ttt{false}}

\newcommand{\upperS}{\ttt{upper}}
\newcommand{\lowerS}{\ttt{lower}}

\newcommand{\lesseq}{\ttt{le}}
\newcommand{\eq}{\ttt{eq}}

\newcommand{\labinfer} [3] [] {\infer[{\small\textsc{#1}}]{#2}{#3}}

\newcommand{\sem} [1] {\llbracket#1\rrbracket}
\newcommand{\Gsem} [1] {\mathcal{G}\sem{#1}}
\newcommand{\Ssem} [1] {\mathcal{S}\sem{#1}}
\newcommand{\Gapprox} {\approx}
\newcommand{\Atoms} {\mathcal{A}}

\newcommand{\partialfn} {\rightharpoonup}
\newcommand{\PZ} {\mathcal{P}(\mathbb{Z})}
\newcommand{\PPZ} {\mathcal{P}(\PZ)}

\newcommand{\sqle} [1] {\sqsubseteq_{#1}}
\newcommand{\boolLE} {\sqle{\bool}}
\newcommand{\intLE} {\sqle{\intS}}
\newcommand{\intsetLE} {\sqle{\intset}}

\newcommand\scalemath[2]{\scalebox{#1}{\mbox{\ensuremath{\displaystyle #2}}}}
\renewcommand{\arraystretch}{0.5}

\begin{document}

\section{Introduction}



% In this document, we describe the semantics of fun-SuSLik and translate them
% into SuSLik's semantics. Then, we translate both fun-SuSLik and SuSLik into a
% category of abstract stores and mappings between of abstract stores. An abstract store is a
% mapping from locations to values of a base type or a tuple of base types (here, the base types are \verb|int| and \verb|bool|).
% By ``location,'' we mean an abstract location: it is not represented as a number and therefore
% you cannot perform pointer arithmetic on it. We do this to avoid these extra details, which
% are not relevant to proving semantics preservation.
%
% An overview of this is given in the diagram below.
% \\
%
% \begin{tikzcd}
%   \text{fun-SuSLik} \arrow[d, "\Gsem{\cdot}"] \arrow[r, swap, "\sigma_{\mathcal{FS}}"]
%   &
%   \text{Abstract stores} \arrow[dddl, "id"]
%   \\
%   \text{Ghost approximation} \arrow[d, "\Ssem{\cdot}"]
%   \\
%   \text{SuSLik} \arrow[d, "\sigma_{\mathcal{S}}"]
%   \\
%   \text{Abstract stores}
%   % \\
%   % \text{Abstract stores}
% \end{tikzcd}
%
% ~\\
% \noindent
% We show that the translation of fun-SuSLik into SuSLik is semantics preserving by proving that this diagram commutes.
% \\
%
% \noindent
% The four categories involved are:
%
% \begin{itemize}
%   \item The fun-SuSLik category of fun-SuSLik values and fun-SuSLik functions
%   \item The ghost approximation category of SuSLik ghost values and functions between them. These ghost values are the values which are manipulated in the pure part of a SuSLik assertion
%   \item The SuSLik category of heaps and SuSLik functions
%   \item The abstract stores category of abstract stores and mappings between abstract stores
% \end{itemize}
%
%

\section{fun-SuSLik Language Definition}
\subsection{Syntax}

\begin{grammar}
  <I> ::= ... | -1 | 0 | 1 | ...

  <B> ::= \ttt{True} | \ttt{False}

  <D> ::= \ttt{x :} <T>; x <P> \ttt{:=} <e>;

  <P> ::= $\varepsilon$ | x <P>

  <e> ::= <e>$_0$ | <e> <e>$_0$

  <e$_0$> ::= <I> | <B> | \ttt{unit} | <L> | <C> | \letbnd\; x \ttt{:=} <e> \inexpr\; <e> | ( <e> )

  <L> ::= $\lambda x \ra$ <e>

  <C> ::= \head\; | \tail\; | \cons\; | \nil\; | \foldr\; | \pair\; | \fst\; | \snd\; | \ite\; | \lesseq\; | \eq

  <T> ::= \ttt{Int} | \ttt{Bool} | \ttt{Unit} | <T> $\ra$ <T> | \ttt{List} <T> | \ttt{Pair} <T> <T> | ( <T> )

  <$\Gamma$> ::= $\cdot$ | x : <T>, $\Gamma$
\end{grammar}

\subsection{Typing Judgment}

\[
  \begin{array}{c}[t]
    \infer{\Gamma \vdash x : T}{(x : T) \in \Gamma}
    ~~~
    \infer{\vdash i : \ttt{Int}}{i \in \mathbb{Z}}
    \\\\
    \infer{\vdash \ttt{unit} : \ttt{Unit}}{}
    ~~~
    \infer{\vdash \ttt{True} : \ttt{Bool}}{}
    ~~~
    \infer{\vdash \ttt{False} : \ttt{Bool}}{}
    \\\\
    \infer{\Gamma \vdash t_0\; t_1 : T'}{\Gamma \vdash t_0 : T \ra T' & \Gamma \vdash t_1 : T}
    \\\\
    \infer{\Gamma \vdash \lambda x \ra t : T \ra T'}{x : T, \Gamma \vdash t : T'}
    \\\\
    \infer{\Gamma \vdash \letbnd\; x\; \ttt{:=}\; e\; \inexpr\; e' : T'}{\Gamma \vdash e : T & x : T, \Gamma \vdash e' : T'}
    \\\\
    \infer{\vdash \ttt{nil} : \List\; T}{}
    ~~~
    \infer{\vdash \ttt{cons} : T \ra \List\; T \ra \List\; T}{}
    \\\\
    \infer{\vdash \ttt{head} : \List\; T \ra T}{}
    ~~~
    \infer{\vdash \ttt{tail} : \List\; T \ra \List\; T}{}
    \\\\
    \infer{\vdash \ttt{foldr} : (T \ra T' \ra T') \ra T' \ra \List\; T \ra T'}{}
    \\\\
    \infer{\vdash \ttt{pair} : T \ra T' \ra \tyPair\; T\; T'}{}
    ~~~
    \infer{\vdash \ttt{fst} : \tyPair\; T\; T' \ra T}{}
    ~~~
    \infer{\vdash \ttt{snd} : \tyPair\; T\; T' \ra T'}{}
    \\\\
    \infer{\vdash \ttt{ifThenElse} : \Bool \ra T \ra T \ra T}{}
    \\\\
    \infer{\vdash \ttt{le} : \Int \ra \Int \ra \Bool}{}
    ~~~
    \infer{\vdash \ttt{eq} : \Int \ra \Int \ra \Bool}{}
  \end{array}
\]

\subsection{Small-Step Operational Semantics}

First, the ``value'' judgment val (as in Harper's Practical Foundations for Programming Languages, Sec. 19.2):

\[
  \begin{array}{cc}
    \infer{i \val}{i \in \mathbb{Z}}
    ~~~
    \infer{\unit \val}{}
    ~~~
    \infer{\True \val}{}
    ~~~
    \infer{\False \val}{}
    ~~~
    \infer{\nil \val}{}
    \\\\
    \infer{\cons\; e_1\; e_2 \val}{e_1 \val & e_2 \val}
    ~~~
    \infer{\pair\; e_1\; e_2 \val}{e_1 \val & e_2 \val}
    ~~~
    \infer{\lambda x \ra e \val}{}
  \end{array}
\]

\noindent
Now the small-step reduction relation $\reduces$


\[
  \begin{array}{cc}
    \labinfer[App-1]{e_1\; e_2 \reduces e_1\; e_2'}{e_2 \reduces e_2'}
    ~~~
    \labinfer[App-2]{e_1\; e_2 \reduces e_1'\; e_2}{e_1 \reduces e_1' & e_2 \val}
    ~~~
    \labinfer[App-$\lambda$]{(\lambda x \ra e)\; e_2 \reduces e[x\mapsto e_2]}{e_2 \val}
    \\\\
    \labinfer[Let-Bnd]{\letbnd\; x\; \ttt{:=}\; e_1\; \inexpr\; e_2 \reduces \letbnd\; x\; \ttt{:=}\; e_1'\; \inexpr\; e_2}
       {e_1 \reduces e_1'}
    \\\\
    \labinfer[Let-Subst]{\letbnd\; x\; \ttt{:=}\; e_1\; \inexpr\; e_2 \reduces e_2[x\mapsto e_1]}{e_1 \val}
    \\\\
    \labinfer[Cons-1]{\cons\; e_1\; e_2 \reduces \cons\; e_1'\; e_2}{e_1 \reduces e_2}
    ~~~
    \labinfer[Cons-2]{\cons\; e_1\; e_2 \reduces \cons\; e_1\; e_2'}{e_1 \val & e_2 \reduces e_2'}
    \\\\
    \labinfer[Head-1]{\head\; e \reduces \head\; e'}{e \reduces e'}
    ~~~
    \labinfer[Tail-1]{\tail\; e \reduces \tail\; e'}{e \reduces e'}
    \\\\
    \labinfer[Head-Cons]{\head\; (\cons\; e_1\; e_2) \reduces e_1}{\cons\; e_1\; e_2 \val}
    ~~~
    \labinfer[Tail-Cons]{\tail\; (\cons\; e_1\; e_2) \reduces e_2}{\cons\; e_1\; e_2 \val}
    \\\\
    \labinfer[Pair-1]{\pair\; e_1\; e_2 \reduces \pair\; e_1'\; e_2}{e_1 \reduces e_1}
    ~~~~
    \labinfer[Pair-2]{\pair\; e_1\; e_2 \reduces \pair\; e_1\; e_2'}{e_1 \val & e_2 \reduces e_2'}
    \\\\
    \labinfer[Foldr-1]{\foldr\; f\; e_0\; e \reduces \foldr\; f'\; e_0\; e}{f \reduces f'}
    ~~~
    \labinfer[Foldr-2]{\foldr\; f\; e_0\; e \reduces \foldr\; f\; e_0'\; e}{f \val & e_0 \reduces e_0'}
    \\\\
    \labinfer[Foldr-3]{\foldr\; f\; e_0\; e \reduces \foldr\; f\; e_0\; e'}{f \val & e_0 \val & e \reduces e'}
    \\\\
    \labinfer[Foldr-Cons]{\foldr\; f\; e_0\; (\cons\; e_1\; e_2) \reduces f\; e_1\; (\foldr\; f\; e_0\; e_2)}{f \val & e_0 \val & \cons\; e_1\; e_2 \val}
    % \\\\
    % % \infer{\ttt{scanr f z $e$} \reduces \ttt{cons e$'$ (f x
    \\\\
    \labinfer[Fst-1]{\fst\; e \reduces \fst\; e'}{e \reduces e'}
    ~~~
    \labinfer[Snd-1]{\snd\; e \reduces \snd\; e'}{e \reduces e'}
    \\\\
    \labinfer[Fst-Pair]{\fst\; (\pair\; e_1\; e_2) \reduces e_1}{\pair\; e_1\; e_2 \val}
    ~~~
    \labinfer[Snd-Pair]{\snd\; (\pair\; e_1\; e_2) \reduces e_2}{\pair\; e_1\; e_2 \val}
    \\\\
    \labinfer[IfThenElse-1]{\ite\; b\; e_1\; e_2 \reduces \ite\; b'\; e_1\; e_2}{b \reduces b'}
    \\\\
    \labinfer[IfThenElse-True]{\ite\; \True\; e_1\; e_2 \reduces e_1}{}
    ~~~
    \labinfer[IfThenElse-False]{\ite\; \False\; e_1\; e_2 \reduces e_2}{}
    \\\\
    \labinfer[Le-1]{\lesseq\; e_1\; e_2 \reduces \lesseq\; e_1'\; e_2}{e_1 \reduces e_1'}
    ~~~
    \labinfer[Le-2]{\lesseq\; e_1\; e_2 \reduces \lesseq\; e_1\; e_2'}{e_1 \val & e_2 \reduces e_2'}
    \\\\
    \labinfer[Le-<]{\lesseq\; e_1\; e_2 \reduces \True}{e_1 \val & e_2 \val & e_1 < e_2}
    ~~~
    \labinfer[Le-=]{\lesseq\; e\; e \reduces \True}{e \val}
    \\\\
    \labinfer[Le-False]{\lesseq\; e_1\; e_2 \reduces \False}{e_1 \val & e_2 \val & e_1 > e_2}
    \\\\
    \labinfer[Eq-1]{\eq\; e_1\; e_2 \reduces \eq\; e_1'\; e_2}{e_1 \reduces e_1'}
    ~~~
    \labinfer[Eq-1]{\eq\; e_1\; e_2 \reduces \eq\; e_1\; e_2'}{e_1 \val & e_2 \reduces e_2'}
    \\\\
    \labinfer[Eq-True]{\eq\; e\; e \reduces \True}{e \val}
    ~~~
    \labinfer[Eq-False]{\eq\; e_1\; e_2 \reduces \False}{e_1 \val & e_2 \val & e_1 \neq e_2}

  \end{array}
\]

\noindent
Let $\reducestr$ be the transitive, reflexive closure of $\reduces$.

\begin{theorem}[Progress]
  $\forall e.\; e \val \lor (\exists e'.\; e \reduces e')$
\end{theorem}

\begin{theorem}[Preservation]
  \text{If} $e : T$ \text{and} $e \reduces e'$ \text{then} $e' : T$
\end{theorem}

\begin{theorem}[Uniqueness of Normal Forms]
  \text{If} $e : T$, $v \val$, $v' \val$, $e \reducestr v$ \text{and} $e \reducestr v'$ \text{then} $v = v'$
\end{theorem}

\noindent
We now define equivalence of fun-SuSLik programs.

\begin{definition}[Termination]
  $e$ \text{is \textnormal{terminating} iff} $\exists v.\; v \val \land e \reducestr v$.\\
  $e$ \text{is \textnormal{nonterminating} iff it is not terminating}
\end{definition}

\begin{definition}[fun-SuSLik Equivalence]
  If $e$ and $e'$ are both nonterminating, then $e \approx_F e'$.\\
  If $e$ and $e'$ are both terminating, then $e \approx_F e'$ iff $\exists v. e \reducestr v \land e' \reducestr v$\\
  Otherwise, $e$ and $e'$ are not equivalent and we write $e \not\approx_F e'$
\end{definition}

\section{SuSLik Assertion Language}

We define the syntax of the SuSLik assertion language and then an equivalence of assertions.

\begin{grammar}
  <n> ::= 0 | 1 | ...

  <i> ::= ... | -1 | 0 | 1 | ...

  <B> ::= \true | \false

  <IS> ::= \{\} | \{ <IL> \} | <IS> -{}- <IS> | <IS> ++ <IS>

  <IL> ::= <i> | <i> , <IL>

  <H> ::= emp | x :-> <t> | [x, <n>]

  <t> ::= x | <i> | <IS> | ( <t> ) | <P> | <t> + <t> | <t> - <t> | <t> * <t>

  <P> ::= <B> | <t> == <t> | <t> $\le$ <t> | not <P> | <P> \&\& <P> | <P> $||$ <P> | <P> ? <t> : <t>

  <S> ::= <H> | <H> * <S>

  <A> ::= \{ <P> ; <S> \}

  <SP> ::= <A> $\leadsto$ <A>
\end{grammar}

\noindent
\begin{definition}[SuSLik Statement Equivalence]
  Given SuSLik statements $c$ and $c'$, $c \approx_S c'$ iff
    $\forall \mathcal{M}. \exists \mathcal{M'}.
      ((c, \mathcal{M}) \BigStep \mathcal{M'})
      \land
      ((c', \mathcal{M}) \BigStep \mathcal{M'})$
\end{definition}

\begin{definition}[SuSLik Specification Equivalence]
  Given SuSLik specifications $S$ and $S'$, $a \approx_{SP} a'$ iff
    $(\forall c\; c'. ((\vdash S | c) \land (\vdash S' | c')) \implies c \approx_{S} c')$
\end{definition}

\section{Translation of fun-SuSLik into SuSLik}

We give a denotational semantics for fun-SuSLik, where the denotations are SuSLik specifications, with
the function $\sem{\cdot} : \ttt{E} \ra \ttt{SP}$.
This translation is shown to preserve equivalence in two steps:

\begin{itemize}
  \item Equivalent fun-SuSLik expressions are mapped to equivalent SuSLik specifications. This amounts to proving
    \textit{full abstraction} for $\sem{\cdot}$.
  \item Non-equivalent fun-SuSLik expressions are mapped to non-equivalent SuSLik specifications.
    This is accomplished by proving $\sem{\cdot}$ is \textit{adequate}.
\end{itemize}

\noindent
More specifically, we have:

\begin{theorem}[Full Abstraction]
  If $e$ and $e'$ are fun-SuSLik expressions and $e \approx_F e'$, then $\sem{e} \approx_{SP} \sem{e'}$
\end{theorem}

\begin{theorem}[Adequacy]
  If $e$ and $e'$ are fun-SuSLik expressions and $e \not\approx_F e'$, then $\sem{e} \not\approx_{SP} \sem{e'}$
\end{theorem}





% \section{Ghost Approximation}
%
% Each subexpression will be assigned a SuSLik ghost variable. This is to specify
% that results are passed from one heaplet to another. This cannot be done directly
% through reusing memory locations across heaplets, since this is not allowed by
% separating conjunction.
%
% % The function $G : (\text{Ctx} \times \text{Expr}) \ra \text{Name}$ maps subexpressions into their corresponding ghost variable.
%
% There are three kinds of ghost variables: \verb|bool|, \verb|int| and \verb|set|. The latter
% represents sets of \verb|int|s. A lambda at the top-level of a program cannot be translated into
% SuSLik. However, lambdas are allowed at subexpressions as long as there are no redexes. Redexes will
% be $\beta$-reduced before applying the following translation.
% \\
%
% \noindent
% \textit{Preconditions for an expression to be translated into SuSLik ghost variables}:
% \begin{itemize}
%   \item No $\beta$-redexes
%   \item No nested structures, since we only have \verb|intset| ghost variables,
%     and no ghost variables which can reflect these more complicated structures. This requirement might be either lifted or relaxed later, depending on
%     if we can find a way to extend SuSLik's capabilities or a way to mitigate this limitation.\\
%     One way could be to provide SuSLik with a program
%     sketch which directly implements the passing of results from one procedure to another. In order for this to work, a SuSLik bug that limits
%     its symbolic execution capabilities must be fixed.
% \end{itemize}
%
% \noindent
% This translation is necessary, as SuSLik uses these ghost variables to specify particular computations. Note that when we translate to
% \verb|intset|s, we forget certain things about the original list (in particular, the ordering of the elements and whether there are "extra"
% elements). We will now describe this translation, and these limitations, in more detail.
%
% In effect, this restriction to \verb|intset|s \textit{requires} that we perform a kind of abstract interpretation of fun-SuSLik into the
% abstract domain of SuSLik ghost values in order to arrive at an appropriate notion of preservation of semantics when we
% translate from fun-SuSLik into SuSLik.
%
%
% \noindent
% The general pipeline of the semantics translation looks like this:
%
% \[
%   {\scriptstyle \text{Operational Semantics of fun-SuSLik} \ra \text{Approximation by SuSLik ghost values} \ra \text{Semantics of SuSLik}}
% \]
%
% First, we will define a language of SuSLik ghost types and ghost values. This will correspond to the pure part of a SuSLik specification.
%
% \textit{A note on notation}: $\mathcal{P}(X)$ is taken to mean the set of all \textit{finite} subsets of $X$. For brevity, we simply write $\mathcal{P}$ instead of 
% $\mathcal{P}_{\textnormal{fin}}$.
%
% \textbf{TODO: Should we give a syntax for sets here that differs from the typical mathematical syntax, to make the distinction more clear?}
% \\
% \begin{grammar}
%   <GT> ::= \bool | \pred | $\PZ$ | $\PPZ$
%
%   <GI> ::= ... | -1 | 0 | 1 | ...
%
%   <GB> ::= \false | \true | <GE> $\le$ <GE> | <GB> \ttt{\&\&} <GB> | <GB> \ttt{||} <GB> | \ttt{not}\; <GB>
%
%   <GS> ::= x | \{\} | <GS> \ttt{++} <GS> | <GS> \ttt{-{}-} <GS> | \{ <GL> \}
%
%   <GL> ::= <GE> | <GE> , <GL>
%
%   <GE> ::= x | <GB> | <GI> | <GE> == <GE> | <GE> + <GE> | <GE> - <GE> | <GE> * <GE> | ( <GE> ) | <GS> | \upperS\; <GS> | \lowerS\; <GS>
% \end{grammar}
%
% We will then define the translation from fun-SuSLik semantics to ghost semantics by the following 3 step process:
%
% \begin{enumerate}
%   \item Define a denotational semantics for the ghost language, $\Gsem{\cdot}$
%   \item Define a transformation $\alpha(\cdot)$ from the fun-SuSLik language of the previous section into this ghost language. We shall do this by defining a mapping of fun-SuSLik values to ghost values and a mapping from fun-SuSLik
% primitive operations into primitive ghost operations
%   \item Require that $\alpha(\cdot)$ be a homomorphism
% \end{enumerate}
%
% \subsection{Typing Judgment}
% \[
%   \begin{array}{cc}
%     \labinfer{\{i\} : \PZ}{i \in \mathbb{Z}}
%     ~~~
%     \labinfer{\true : \bool}{}
%     ~~~
%     \labinfer{\false : \bool}{}
%     ~~~
%     \labinfer{\{\} : \PPZ}{}
%     \\\\
%     \labinfer{x\; \ttt{++}\; y : \PPZ}{x : \PPZ & y : \PPZ}
%     ~~~
%     \labinfer{x\; \ttt{-{}-}\; y : \PPZ}{x : \PPZ & y : \PPZ}
%
%   \end{array}
% \]
%
% \subsection{Big-Step Operational Semantics}
%
% \textbf{TODO: Write}
%
% % \[
% %   \begin{array}{c}
% %     \labinfer{\{i\} \Downarrow \{i\}}{i \in \mathbb{Z}}
% %   \end{array}
% % \]
%
% \subsection{Translation on Types}
%
% The action of $\alpha(\cdot)$ on types is given by
%
% \[
%   \begin{array}{cc}
%     \alpha(\Int) = \PZ
%     ~~~
%     \alpha(\Bool) = \bool
%     ~~~
%     \alpha(\List\; \Int) = \PPZ
%     ~~~
%     \alpha(A \ra B) = \pred
%   \end{array}
% \]
%
% Note that $\List\; \Int$ is represented by sets of sets of integers and \textbf{nested lists are not currently supported}.
%
% \subsection{Partial Orders}
%
% We define three partial orders: one for each ghost type (other than $\pred$). For \verb|bool|, the relation
% will simply be the equality relation (the ``flat'' partial order). For \verb|int| and \verb|intset|, it will be
% the subset relation.
%
% \[
%   \begin{array}{cc}
%     \labinfer{x \boolLE x}{x : \bool}
%     ~~~
%     \labinfer{x \intLE y}{x : \intS & y : \intS & x \subseteq y}
%     % \sqsubseteq_{\intS}\; = \{ (x, x) | x : \intS \}
%     % \sqsubseteq_{\bool}\; = \{ (x, x) | x : \bool \}
%     ~~~
%     \labinfer{x \intsetLE y}{x : \intset & y : \intset & x \subseteq y}
%     % \sqsubseteq_{\intset}\; = \{ (x, y) | x : \intset, y : \intset, x \subseteq y \}
%   \end{array}
% \]
%
% \subsection{Interpreting fun-SuSLik in the Ghost Semantics}
%
% % First, we require our interpretation function, $\Gsem(\cdot)$, to respect the partial orders described in the previous
% % section. We express this as the law:
% %
% % \[
% %   \labinfer{\Gsem{e} 
% % \]
% %
% % Now, we actually define the function as follows:
%
% As not all fun-SuSLik constructs have a representation in the ghost language (the pure part),
% we use a partial function $\Gsem{\cdot} : Expr \partialfn \mathcal{G}$.
%
% \[\arraycolsep=1.4pt\def\arraystretch{1.4}
%   \begin{array}{c}
%     \Gsem{i} \ni i~~~\text{where $i \in \mathbb{Z}$}
%     \\
%     \Gsem{\True} \boolLE \true
%     \\
%     \Gsem{\False} \boolLE \false
%     \\
%     \Gsem{\nil} \intsetLE \emptyset
%     \\
%     \Gsem{\cons\; e_1\; e_2} \intsetLE \{ \Gsem{e_1} \} \cup \Gsem{e_2}
%     \\
%     \Gsem{\head\; e} \intLE \bigcup\Gsem{e}
%     \\
%     \Gsem{\tail\; e} \intsetLE \{ s \setminus x\; |\; s \in \Gsem{e}, x \in \Gsem{\head\; e}\}
%     \\
%     \Gsem{\lambda x \ra e} = \pred(x)\; \ttt{=>}\; \Gsem{e}
%     \\
%     \Gsem{(\lambda x \ra e)\; e_2} \sqle{\alpha(B)} \Gsem{e[x\mapsto e_2]}~~~\text{where $\lambda\; x \ra e : A \ra B$}
%     \\
%     \Gsem{e_1\; e_2} = \Gsem{e_1}(\Gsem{e_2})~~~\text{where $e_1$ is not a lambda}
%     % \Gsem{\foldr\; (\lambda x \ra e)\; e_0\; e_1} \in { 
%   \end{array}
% \]
%
%
%
% % ~\\\\
% % \textbf{TODO: Rewrite the following part to fit better with the preceding part}
% %
% % \noindent
% % The semantic function which translate fun-SuSLik values is given by
% % \[
% %   \Gsem{\cdot} : Expr \ra \mathcal{P}(\mathbb{Z} \uplus \mathcal{P}(\mathbb{Z}))
% % \]
% %
% % \noindent
% % Note that $\mathcal{P}(\mathbb{Z})$ is the set of all possible values of type \verb|intset|.\\
% %
% % The image of a fun-SuSLik expression in $\Gsem{\cdot}$ is the
% % set of all possible ghost values that can represent it. For example, for a fun-SuSLik list \verb|cons 1 (cons 2 (cons 3 nil))|, the possible
% % ghost values would be all \verb|intset|s that contain \ttt{1}, \ttt{2} and \ttt{3}. Compare this with the fun-SuSLik expression
% % \verb|head e|. After translating this to ghost values, all we know is that the result will be some value in the
% % list \verb|e|. By translating into sets, we've forgotten exactly which one it will be.
% %
% % We are particularly interested in minimal \textit{non-empty} elements in the image of
% % $\Gsem{\cdot}$. Again, consider the case of $\Gsem{\cons\; 1\; (\cons\; 2\; (\cons\; 3\; \nil))}$.
% % This will translate to the set of all \verb|intset| values that are supersets of $\{1,2,3\}$. However, there we notice that one
% % element of this set is most "important": the set $\{1,2,3\}$ itself. This set corresponds
% % \textit{exactly} to the elements of the original list, without containing any
% % "extra" elements.
% %
% % Also notice that, if you consider the $\subseteq$ partial order,
% % this set is a \textit{minimal} set in $\Gsem{\cons\; 1\; (\cons\; 2\; (\cons\; 3\; \nil))}$. In particular,
% % the greatest lower bound of $\Gsem{\cons\; 1\; (\cons\; 2\; (\cons\; 3\; \nil))}$ is $\{1,2,3\}$. In mathematical notation,
% % we can write:
% %
% % \[
% %   \bigcap\Gsem{\cons\; 1\; (\cons\; 2\; (\cons\; 3\; \nil))} = \{1, 2, 3\}
% % \]
% %
% % Further, note that there are cases where the greatest lower bound is just $\emptyset$. This is the case for \verb|head e|, as mentioned
% % above. To see this, observe that $\{\{1\},\{2\},\{3\}\} \subseteq \Gsem{\head\; (\cons\; 1\; (\cons\; 2\; (\cons\; 3\; \nil)))}$. Therefore,
% % we see that $\bigcap{\Gsem{\head\; (\cons\; 1\; (\cons\; 2\; (\cons\; 3\; \nil)))}} = \emptyset$.
% %
% % Thus, we are more interested in the \textit{atoms} in the image of $\Gsem{\cdot}$. An atom is any minimal element that is
% % strictly greater than the least element (the least element is $\emptyset$ in our case). Continuing our example, the atoms of $\Gsem{\head\; (\cons\; 1\; (\cons\; 2\; (\cons\; 3\; \nil)))}$ are $\{1\}$, $\{2\}$ and $\{3\}$. Furthermore, the unique atom of $\Gsem{\cons\; 1\; (\cons\; 2\; (\cons\; 3\; \nil))}$ is $\{1,2,3\}$.
% %
% % We can see that we do not just want to look at the minimal elements, we want to look at the \textit{atoms}. For this, we use the
% % following notation: Let $\Atoms X$ be the set of all atoms in $X$. Using this for our above example, we see that while the greatest lower bound
% % gives us this
% %
% % \[
% %   \bigcap\Gsem{\head\; (\cons\; 1\; (\cons\; 2\; (\cons\; 3\; \nil)))} = \emptyset
% % \]
% %
% % \noindent
% % if we instead use $\Atoms$, we get
% %
% % \[
% %   \Atoms\Gsem{\head\; (\cons\; 1\; (\cons\; 2\; (\cons\; 3\; \nil)))} = \{\{1\},\{2\},\{3\}\}
% % \]
% %
% % \noindent
% % The general list of rules for $\Gsem{\cdot}$ are given here:
% %
% % % Consider G[e] = {{1,2,3}, {1,2,3,4}, {1,2,3,4,5}, ...}
% % % and we want to find G[head e] = {{1}, {2}, {3}, ...}
% %
% % \[
% %   \begin{array}{cc}
% %     \Gsem{\nil} = \{\emptyset\}
% %     ~~~~~
% %     \Gsem{\cons\; e_1\; e_2} = \{ \Atoms\Gsem{e_1} \cup a\; |\; a \in \Atoms\Gsem{e_2} \}
% %     \\
% %     \Gsem{\head\; e} = \{ \Atoms s\; |\; s \in \Atoms\Gsem{e}\ \}
% %     ~~~~~
% %     \Gsem{\tail\; e} = \Gsem{e} \cap \Atoms\Gsem{e}
% %     % \Atoms\Gsem{\cons\; e_1\; e_2} = \{ \Atoms\Gsem{e_1} \} \cup \Atoms\Gsem{e_2}
% %   \end{array}
% % \]
% %
% %
% % \section{Translation Into SuSLik}
% %
% %
% % \noindent
% % The translation into SuSLik will be given by the semantic function $\Ssem{\cdot}$
% %
% % \noindent
% % \textbf{TODO:} Rewrite this to use the ghost variable approximation
% %
% % \[
% %   \begin{array}{cc}
% %     \Ssem{\letbnd\; x\; \ttt{:=}\; e\; \inexpr\; e_2} = x \mapsto v * \Ssem{e_2}(v) * \Ssem{e}(v)~~~\text{where $v \notin FV(\Ssem{e}) \cup FV(\Ssem{e_2})$}
% %     \\
% %     \Ssem{\lambda\; x\; \ra\; e} = \text{pred}(x) \Ra \Ssem{e}
% %     % \\
% %     % \sem{\cons\; e_1\; e_2} = \text{cons}(v
% %   \end{array}
% % \]
%

\end{document}


\documentclass[10pt]{article}
\usepackage{amsmath, listings, amsthm, amssymb, latexsym, proof, syntax, stmaryrd, tikz-cd}
\usepackage[T1]{fontenc}
\usepackage{scrlayer-scrpage,hyperref}

\newtheorem{theorem}{Theorem}
\newtheorem{definition}{Definition}

\lstset{
  % basewidth=0.585em,
  basicstyle=\ttfamily,
  mathescape=true,
  literate=
    {>->}{$\rightarrowtail$}{1}
    {:->}{$:\rightarrow$}{1}
    {->}{$\rightarrow$}{1}
    {<-}{$\leftarrow$}{1}
    % {lift}{$\uparrow$}{1}
    % {lower}{$\downarrow$}{1}
    {ell}{$\ell$}{1}
}

% \renewcommand{\syntleft}{}
% \renewcommand{\syntright}{}

\newcommand{\infers}{\ensuremath{\Rightarrow}}
\newcommand{\checks}{\ensuremath{\Leftarrow}}

\newcommand{\ttt}[1]{\texttt{#1}}
\newcommand{\Tlst}[1]{\text{\lstinline{#1}}}
\newcommand{\lst}[1]{\lstinline{#1}}
\newcommand{\ra}{\ensuremath{\rightarrow}}
\newcommand{\Ra}{\ensuremath{\Rightarrow}}
\newcommand{\hra}{\ensuremath{\rightarrowtail}}

\newcommand{\la}{\ensuremath{\leftarrow}}

\newcommand{\reduces}{\ensuremath{\longmapsto}}
\newcommand{\reducestr}{\ensuremath{\longmapsto^{*}}}

\newcommand{\judgecolor}{\color{red}}

\newcommand{\bigs}{{\judgecolor\;{\Downarrow}}}
\newcommand{\Lbigs}{{\judgecolor\;\mathcal{L}\!\!\Downarrow}}
\newcommand{\LbigsGo}{{\judgecolor\;\mathcal{L}_1\!\!\Downarrow}}
\newcommand{\Abigs}{{\judgecolor\;\mathcal{A}\!\!\Downarrow}}

% \newcommand{\layoutPart} [1] {{\color{blue!70!black}\;#1}}
\newcommand{\layoutPart} [1] {{\color{blue}\;#1}\;}

\newcommand{\pointsto}{\ensuremath{\mapsto}}
\newcommand{\sep}{\ensuremath{\,*\,}}

\newcommand{\BigStep}{\ensuremath{\Downarrow}}

\newcommand{\toCtx}{\;{\judgecolor\vartriangleright}\;}

\newcommand{\seq}{\;{\judgecolor\vdash}\;}

\newcommand{\val}{\textnormal{ val}}
\newcommand{\ok}{\textnormal{ ok}}

\newcommand{\True}{\ttt{True}}
\newcommand{\False}{\ttt{False}}
\newcommand{\head}{\ttt{head}}
\newcommand{\tail}{\ttt{tail}}
\newcommand{\cons}{\ttt{cons}}
\newcommand{\nil}{\ttt{nil}}
\newcommand{\unit}{\ttt{unit}}
\newcommand{\foldr}{\ttt{foldr}}
\newcommand{\fold} [1] {\ttt{fold}$_{#1}$}
% \newcommand{\scanr}{\ttt{scanr}}
\newcommand{\pair}{\ttt{pair}}
\newcommand{\fst}{\ttt{fst}}
\newcommand{\snd}{\ttt{snd}}
\newcommand{\ite}{\ttt{if}}
\newcommand{\add}{\ttt{add}}
\newcommand{\sub}{\ttt{sub}}

\newcommand{\Int}{\ttt{Int}}
\newcommand{\Bool}{\ttt{Bool}}
\newcommand{\tyUnit}{\ttt{Unit}}
\newcommand{\tyPair}{\ttt{Pair}}
\newcommand{\List}{\ttt{List}}
\newcommand{\letbnd}{\ttt{let}}
\newcommand{\inexpr}{\ttt{in}}
\newcommand{\defeq}{\ttt{:=}}

\newcommand{\intS}{\ttt{int}}
\newcommand{\intset}{\ttt{set}}
\newcommand{\bool}{\ttt{bool}}
\newcommand{\pred}{\ttt{pred}}

\newcommand{\true}{\ttt{true}}
\newcommand{\false}{\ttt{false}}

\newcommand{\liftS}{\ttt{lift}\;}
\newcommand{\lowerS}{\ttt{lower}}

\newcommand{\lesseq}{\ttt{le}}
\newcommand{\eq}{\ttt{eq}}

\newcommand{\labinfer} [3] [] {\infer[{\small\textsc{#1}}]{#2}{#3}}

\newcommand{\sem} [1] {\llbracket#1\rrbracket}
\newcommand{\Csem} [1] {\mathcal{C}\sem{#1}}
\newcommand{\Isem} [1] {\mathcal{I}\sem{#1}}
\newcommand{\Gsem} [1] {\mathcal{G}\sem{#1}}
\newcommand{\Ssem} [1] {\mathcal{S}\sem{#1}}
\newcommand{\Esem} [1] {\mathcal{E}\sem{#1}}
\newcommand{\Asem} [2] {\mathcal{A}_{#1}\sem{#2}}
\newcommand{\Fsem} [2] {\mathcal{F}_{#1}\sem{#2}}
\newcommand{\Nsem} [2] {\mathcal{N}_{#1}\sem{#2}}
\newcommand{\Rsem} [2] {\mathcal{R}_{#1}\sem{#2}}
\newcommand{\Tsem} [2] {\mathcal{T}_{#1}\sem{#2}}
\newcommand{\Lsem} [1] {\mathcal{L}\sem{#1}}
\newcommand{\Gapprox} {\approx}
\newcommand{\Atoms} {\mathcal{A}}

\newcommand{\partialfn} {\rightharpoonup}
\newcommand{\PZ} {\mathcal{P}(\mathbb{Z})}
\newcommand{\PPZ} {\mathcal{P}(\PZ)}

\newcommand{\sqle} [1] {\sqsubseteq_{#1}}
\newcommand{\boolLE} {\sqle{\bool}}
\newcommand{\intLE} {\sqle{\intS}}
\newcommand{\intsetLE} {\sqle{\intset}}

\newcommand\scalemath[2]{\scalebox{#1}{\mbox{\ensuremath{\displaystyle #2}}}}
\renewcommand{\arraystretch}{0.5}

\newcommand{\fresh}{\ttt{fresh}}
\newcommand{\Name}{\ttt{Name}}

\newcommand{\bind}{>\!\!\!>\!\!=}

\newcommand{\first}{proj_1}
\newcommand{\second}{proj_2}

\newcommand{\blank}{\cdot}
\newcommand{\alt}{\,|\,}

\newcommand{\synth}{\;|\;}

\newcommand{\angled} [1] {\langle #1 \rangle}

\newcommand{\Alts}{\overline{\textnormal{A}}}

\newcommand{\reducesA}{\reduces_{\boldsymbol{a}}}

\renewcommand{\syntleft}{}
\renewcommand{\syntright}{}

\newcommand{\compile}{\textnormal{compile}}
\newcommand{\uncompile}{\textnormal{uncompile}}

\endinput



\begin{document}

\section{Introduction}

In this document, we give a translation from fun-SuSLik to SuSLik. First, we define the syntax and
operational semantics of the two languages (Sections~\ref{sec:FSDef} and~\ref{sec:SuSLikDef}).
Then, we give definitions of semantic equivalence in terms of these
operational semantics (Sections~\ref{sec:FSEquiv} and~\ref{sec:SuSLikEquiv}). The translation from fun-SuSLik is represented as a denotational semantics
that denotes a fun-SuSLik expression as a SuSLik goal (Section~\ref{sec:Translation}). Finally, we show that this denotational
semantics preserves semantic equivalence by proving adequacy and full abstraction (Section~\ref{sec:TransSoundness}).
\\


% In this document, we describe the semantics of fun-SuSLik and translate them
% into SuSLik's semantics. Then, we translate both fun-SuSLik and SuSLik into a
% category of abstract stores and mappings between of abstract stores. An abstract store is a
% mapping from locations to values of a base type or a tuple of base types (here, the base types are \verb|int| and \verb|bool|).
% By ``location,'' we mean an abstract location: it is not represented as a number and therefore
% you cannot perform pointer arithmetic on it. We do this to avoid these extra details, which
% are not relevant to proving semantics preservation.
%
% An overview of this is given in the diagram below.
% \\
%
% \begin{tikzcd}
%   \text{fun-SuSLik} \arrow[d, "\Gsem{\cdot}"] \arrow[r, swap, "\sigma_{\mathcal{FS}}"]
%   &
%   \text{Abstract stores} \arrow[dddl, "id"]
%   \\
%   \text{Ghost approximation} \arrow[d, "\Ssem{\cdot}"]
%   \\
%   \text{SuSLik} \arrow[d, "\sigma_{\mathcal{S}}"]
%   \\
%   \text{Abstract stores}
%   % \\
%   % \text{Abstract stores}
% \end{tikzcd}
%
% ~\\
% \noindent
% We show that the translation of fun-SuSLik into SuSLik is semantics preserving by proving that this diagram commutes.
% \\
%
% \noindent
% The four categories involved are:
%
% \begin{itemize}
%   \item The fun-SuSLik category of fun-SuSLik values and fun-SuSLik functions
%   \item The ghost approximation category of SuSLik ghost values and functions between them. These ghost values are the values which are manipulated in the pure part of a SuSLik assertion
%   \item The SuSLik category of heaps and SuSLik functions
%   \item The abstract stores category of abstract stores and mappings between abstract stores
% \end{itemize}
%
%

\section{fun-SuSLik Language Definition}
\label{sec:FSDef}
\subsection{Syntax}

\begin{grammar}
  <$\mathcal{D}$> ::= "data" D "where" $\Alts$;

  <A> ::= "C" : <T>

  <$\mathcal{L}$> ::= "L" : D $\hra$ "layout"["x"]; $\overline{\textnormal{LM}}$

  <LM> ::= "L" ("C" $\overline{"y"}$) ":=" $\overline{\textnormal{H}}$

  <H> ::= "p" ":->" "y"

  <p> ::= "x" | "(" "x" "+" <n> ")"

  <n> ::= 0 | 1 | 2 $\cdots$

  <I> ::= $\cdots$ | -1 | 0 | 1 | $\cdots$

  <B> ::= "True" | "False"

  <D> ::= "f" : <T>; "f" <P> ":=" <e>;

  <P> ::= $\cdot$ | "x" <P>

  <e> ::= <e>$_0$ | <e> <e>$_0$

  <e$_0$> ::= <I> | <B> | $\ell$ | <C> | "let" x ":=" <e> "in" <e> | ( <e> )

  <$\ell$> ::= $\lambda "x" \ra$ <e>

  <C> ::= "convert"$_{"L","L"}$ | "fold" | $\uparrow$ | $\downarrow$ | \ite\; | \lesseq\; | \eq\; | \add\; | \sub\; | \ttt{and}\; | \ttt{or}\; | \ttt{not}

  <T> ::= "Int" | "Bool" | <T> $\ra$ <T> | "L" ( "D" ) | ( <T> )

  <$\Gamma$> ::= $\cdot$ | "x" : <T>, $\Gamma$

  <$\Sigma$> ::= $\overline{\mathcal{D}}$
\end{grammar}

\noindent
For convenience, we define the function for getting the type name of a type definition:

\[
  \tau(\ttt{data T = }\Alts) = \ttt{T}
\]

and the function for getting a list of constructor type signatures:

\begin{align*}
  &\sigma(\ttt{data T = ;}) = []
  \\
  &\sigma(\ttt{data T = C : S | }\Alts) = (C : S) :: \sigma({\ttt{data T = }}\Alts)
\end{align*}

Further, we define a set of base types

\[
  \textnormal{base} = \{\Int, \Bool\}
\]

\subsection{fun-SuSLik Examples}

\begin{lstlisting}
data List where
  Nil : List
  Cons : Int -> List -> List

SinglyLinked : List >-> layout[x]
SinglyLinked Nil = emp
SinglyLinked (Cons head tail) =
  x :-> head, (x+1) :-> tail, SinglyLinked tail

DoublyLinked : List >-> layout[x, z]
DoublyLinked Nil = emp
DoublyLinked (Cons head tail) = 
  x :-> head, (x+1) :-> w, (x+2) :-> z, DoublyLinked[w, x] tail

-- "Layout polymorphic" function
sumList : List -> Int
sumList = fold add 0

listExample : List
listExample = Cons 1 (Cons 2 (Cons 3 Nil))

singlyLinkedExample : SinglyLinked
singlyLinkedExample = lower listExample

doublyLinkedExample : DoublyLinked
doublyLinkedExample = lower listExample

sumSinglyLinkedExample : Int
sumSinglyLinkedExample = (lower sumList) singlyLinkedExample

sumDoublyLinkedExample : Int
sumDoublyLinkedExample = (lower sumList) doublyLinkedExample

singlyToDoubly : DoublyLinked
singlyToDoubly = convert singlyLinkedExample
\end{lstlisting}

\subsection{Well-Formedness For Type Definitions}

In a type definition for some new type \ttt{T}, each constructor type signature must have \ttt{T} as its result type. This is
formalized below as the$\ok$ judgment for type definitions. This judgment, in turn, uses the $\ok_A$ judgment for constructor
type signatures. We define$\ok$ on lists of constructor types, then we extend this to type definitions using $\sigma(\cdot)$.

\[
  \begin{array}{c}
    \infer{\mathcal{D} \ok}{\sigma(\mathcal{D}) \ok(\tau(\mathcal{D}))}
    \\\\

    \infer{[] \ok(T)}{}
    ~~~
    \infer{(C : S) :: \overline{a} \ok(T)}{S \ok_A(T) & \overline{a} \ok(T) }
    \\\\
    \infer{\ttt{T }\ok_A(\ttt{T})}{}
    ~~~
    \infer{\ttt{S} \ra \ttt{R} \ok_A(\ttt{T})}{\ttt{R} \ok_A(\ttt{T})}
  \end{array}
\]

\noindent
In short, if you can derive $\mathcal{D}\ok$ for a type definition $\mathcal{D}$ using the rules above, then $\mathcal{D}$ is well-formed.

\subsection{Typing Judgment}

\[
  \begin{array}{c}
    \infer{\Sigma ; \Gamma \vdash e : \ttt{layout}[x]}{\Sigma ; \Gamma \vdash e : D & \Sigma ; \Gamma \vdash L : D \hra \ttt{layout}[x]}
    \\\\
    \infer{\Sigma ; \Gamma \vdash x : T}{(x : T) \in \Gamma}
    ~~~
    \infer{\vdash i : \ttt{Int}}{i \in \mathbb{Z}}
    \\\\
    \infer{\vdash \ttt{True} : \ttt{Bool}}{}
    ~~~
    \infer{\vdash \ttt{False} : \ttt{Bool}}{}
    \\\\
    \infer{\Sigma ; \Gamma \vdash t_0\; t_1 : T'}{\Sigma ; \Gamma \vdash t_0 : T \ra T' & \Sigma ; \Gamma \vdash t_1 : T}
    \\\\
    \infer{\Sigma ; \Gamma \vdash \lambda x \ra t : T \ra T'}{x : T, \Sigma ; \Gamma \vdash t : T'}
    \\\\
    \infer{\Sigma ; \Gamma \vdash \letbnd\; x\; \ttt{:=}\; e\; \inexpr\; e' : T'}{\Sigma ; \Gamma \vdash e : T & x : T, \Sigma ; \Gamma \vdash e' : T'}
    \\\\
    \infer{\Sigma ; \Gamma \vdash \ttt{ifThenElse c t f} : T}{\Sigma ; \Gamma \vdash \ttt{c} : \Bool & \Sigma ; \Gamma \vdash \ttt{t} : T & \Sigma ; \Gamma \vdash \ttt{f} : T}
    \\\\
    \infer{\Sigma ; \Gamma \vdash \ttt{le x y} : \Bool}{\Sigma ; \Gamma \vdash \ttt{x} : \Int & \Sigma ; \Gamma \vdash \ttt{y} : \Int}
    ~~~
    \infer{\Sigma ; \Gamma \vdash \ttt{eq x y} : \Bool}{\Sigma ; \Gamma \vdash \ttt{x} : \Int & \Sigma ; \Gamma \vdash \ttt{y} : \Int}
    \\\\
    \infer{\Sigma ; \Gamma \vdash \ttt{add x y} : \Int}{\Sigma ; \Gamma \vdash \ttt{x} : \Int & \Sigma ; \Gamma \vdash \ttt{y} : \Int}
    ~~~
    \infer{\Sigma ; \Gamma \vdash \ttt{sub x y} : \Int}{\Sigma ; \Gamma \vdash \ttt{x} : \Int & \Sigma ; \Gamma \vdash \ttt{y} : \Int}
    \\\\
    \infer{\Sigma ; \Gamma \vdash \ttt{convert}_{\tau(\mathcal{D}),\tau({\mathcal{D}'})} : \tau(\mathcal{D}) \ra \tau(\mathcal{D}')}{\mathcal{D} \in \Sigma & \mathcal{D}' \in \Sigma}
    \\\\
    \infer{\Sigma ; \Gamma \vdash \ttt{fold}_{\tau(\mathcal{D})} \ttt{ f z} : \tau(\mathcal{D}) \ra S}{(R,\tau(\mathcal{D}),\sigma(\mathcal{D}))\; \mathcal{F}\; S & \Sigma ; \Gamma \vdash f : R & \Sigma ; \Gamma \vdash z : S & \mathcal{D} \in \Sigma}
  \end{array}
\]

\textbf{TODO: Get this to work}
\noindent
The following relation gives the ``type specification'' of a fold, which includes a unification variable $\alpha$. As there is no
polymorphism in fun-SuSLik, this unification variable is only used internally and is not user-visible. It must eventually
unify with a concrete type.



% Where the 4-ary relation defined by the following rules give the type for a fold, of the form $(A, B, C)\; \mathcal{F}\; D$, of a data definition:
% \\
% \textbf{TODO: Use color/bolding to indicate that $T$ always the original type name below?}
%
% \[
%   \begin{array}{c}
%     \infer{(T, T, T)\; \mathcal{F\; } T}{}
%     ~~~
%     \infer{(X \ra Y_1, T, (X \ra Y_2) \ra R)\; \mathcal{F}\; R'}{}
%     % &\mathcal{F}((X \ra Y \ra Z), (X \ra Y \ra Z) \ra R) = (X \ra Y \ra Z) \ra Z
%   \end{array}
% \]

\subsection{Small-Step Operational Semantics}

First, the ``value'' judgment val (as in Harper's Practical Foundations for Programming Languages, Sec. 19.2):

\[
  \begin{array}{cc}
    \infer{i \val}{i \in \mathbb{Z}}
    ~~~
    \infer{\True \val}{}
    ~~~
    \infer{\False \val}{}
    ~~~
    \infer{\nil \val}{}
    \\\\
    \infer{\cons\; e_1\; e_2 \val}{e_1 \val & e_2 \val}
    ~~~
    \infer{\pair\; e_1\; e_2 \val}{e_1 \val & e_2 \val}
    ~~~
    \infer{\lambda x \ra e \val}{}
  \end{array}
\]

\noindent
Now the small-step reduction relation $\reduces$\\

\textbf{$\bullet$ TODO: What would make sense as a rule for reducing \ttt{convert}$_{S,T}$ applications?}

We define a reduction semantics $\mathcal{E}$ where $v$ is an expression such that the judgment $v \val$ holds.

{\footnotesize
\begin{align*}
  \mathcal{E} ::=\\
    &[\,]\\
    &\mathcal{E}\; \angled{e}\\
    &v\; \mathcal{E}\\
    &\letbnd\; x = \mathcal{E}\; \inexpr\; \angled{e}\\
    &\letbnd\; x = v\; \inexpr\; \mathcal{E}\\
    &\ttt{fold } \mathcal{E}\; \angled{e}\; \angled{e}\\
    &\ttt{fold } v\; \mathcal{E}\; \angled{e}\\
    &\ttt{fold } v\; v\; \mathcal{E}\\
    &\ttt{if } \mathcal{E}\; \angled{e}\; \angled{e}\\
    &\ttt{if True } \mathcal{E}\; \angled{e}\\
    &\ttt{if False } \angled{e}\; \mathcal{E}\\
    &\ttt{le } \mathcal{E}\; \angled{e}\\
    &\ttt{le } v\; \mathcal{E}\\
    &\ttt{eq } \mathcal{E}\; \angled{e}\\
    &\ttt{eq } v\; \mathcal{E}\\
    &\ttt{add } \mathcal{E}\; \angled{e}\\
    &\ttt{add } v\; \mathcal{E}\\
    &\ttt{sub } \mathcal{E}\; \angled{e}\\
    &\ttt{sub } v\; \mathcal{E}\\
\end{align*}
}

The reduction relation $\reduces$ is given by:

\[
  \begin{array}{c}
    \labinfer{\mathcal{E}[e] \reduces \mathcal{E}[e']}{e \reducesA e'}
  \end{array}
\]

with $\reducesA$ given by:

\[
  \begin{array}{cc}
    \labinfer[App-$\lambda$]{(\lambda x \ra e)\; e_2 \reducesA e[x\mapsto e_2]}{e_2 \val}
    \\\\
    \labinfer[Let-Subst]{\letbnd\; x\; \ttt{:=}\; e_1\; \inexpr\; e_2 \reducesA e_2[x\mapsto e_1]}{e_1 \val}
    \\\\
    \labinfer[IfThenElse-True]{\ite\; \True\; e_1\; e_2 \reducesA e_1}{}
    ~~~
    \labinfer[IfThenElse-False]{\ite\; \False\; e_1\; e_2 \reducesA e_2}{}
    \\\\
    \labinfer[Le-<]{\lesseq\; e_1\; e_2 \reducesA \True}{e_1 \val & e_2 \val & e_1 < e_2}
    ~~~
    \labinfer[Le-=]{\lesseq\; e\; e \reducesA \True}{e \val}
    \\\\
    \labinfer[Le-False]{\lesseq\; e_1\; e_2 \reducesA \False}{e_1 \val & e_2 \val & e_1 > e_2}
    \\\\
    \labinfer[Eq-True]{\eq\; e\; e \reducesA \True}{e \val}
    ~~~
    \labinfer[Eq-False]{\eq\; e_1\; e_2 \reducesA \False}{e_1 \val & e_2 \val & e_1 \neq e_2}
    \\\\
    \labinfer[Add-Apply]{\add\; e_1\; e_2 \reducesA v}{e_1 \val & e_2 \val & v = e_1 + e_2}

  \end{array}
\]

\noindent
Let $\reducestr$ be the transitive, reflexive closure of $\reduces$.

\begin{theorem}[Progress]
  $\forall e.\; e \val \lor (\exists e'.\; e \reduces e')$
\end{theorem}

\begin{theorem}[Preservation]
  \text{If} $e : T$ \text{and} $e \reduces e'$ \text{then} $e' : T$
\end{theorem}

\begin{theorem}[Uniqueness of Normal Forms]
  \text{If} $e : T$, $v \val$, $v' \val$, $e \reducestr v$ \text{and} $e \reducestr v'$ \text{then} $v = v'$
\end{theorem}

\subsection{Equivalence Relation}
\label{sec:FSEquiv}
We now define equivalence of fun-SuSLik programs.

\begin{definition}[Termination]
  $e$ \text{is \textnormal{terminating} iff} $\exists v.\; v \val \land e \reducestr v$.\\
  $e$ \text{is \textnormal{nonterminating} iff it is not terminating}
\end{definition}

\begin{definition}[fun-SuSLik Equivalence]
  If $e$ and $e'$ are both nonterminating, then $e \approx_F e'$.\\
  If $e$ and $e'$ are both terminating, then $e \approx_F e'$ iff $\exists v. e \reducestr v \land e' \reducestr v$\\
  Otherwise, $e$ and $e'$ are not equivalent and we write $e \not\approx_F e'$
\end{definition}

\section{SuSLik Specification Language}
\label{sec:SuSLikDef}

We define the syntax of the SuSLik specification language and then an equivalence of SuSLik synthesis goals.
The following syntax definition is adapted from \textit{Cyclic Program Synthesis, 2021}:
\\
\[ \small
  \begin{array}[t]{ l l }
    \text{Variable} &x, y\;\; \text{Alpha-numeric identifiers $\in$ PV}\\
    \text{Size, offset} &n, \iota\;\; \text{Non-negative integers}\\
    \text{Expression} &e ::= 0\; \alt \true \alt x \alt e = e \alt e \land e \alt \neg e \alt d\\
    \text{$\mathcal{T}$-expr.} &d ::= n \alt x \alt d + d \alt n \cdot d \alt \{\} \alt {d} \alt \cdots\\
    \text{Command} &c ::= \letbnd\; \ttt{x = *(x + }\iota\ttt{)} \alt \ttt{*(x + }\iota\ttt{) = e} \alt
      \letbnd\; \ttt{x = malloc(n)} \alt \ttt{free(x)} \alt \ttt{error}\newline
      \alt \ttt{f(}\overline{e_i}\ttt{)}\\
    \text{Program} &\Pi ::= \overline{f(\overline{x_i})\; \{\; c\; \}\; ;}\; c\\

    \text{Logical variable} &\nu, \omega\\
    \text{Cardinality variable} &\alpha\\
    \text{$\mathcal{T}$-term} &\kappa ::= \nu \alt e \alt \cdots\\
    \text{Pure logic term} &\phi, \psi, \chi ::= \kappa \alt \phi = \phi \alt \phi \land \phi \alt \neg \phi\\
    \text{Symbolic heap} &P, Q, R ::= \ttt{emp} \alt \mbox{$\langle e, \iota \rangle \mapsto e \alt [e, \iota]$} \alt p^{\alpha}(\overline{\phi_i})
      \alt \mbox{$P * Q$}\\
    \text{Heap predicate} &\mathcal{D} ::= p^{\alpha}(\overline{x_i}) : \overline{e_j \Ra \exists \overline{y}.\{\chi_j;R_j\}}\\
    \text{Assertion} &\mathcal{P},\mathcal{Q} ::= \{\phi; P\}\\
    \text{Environment} &\Gamma ::= \forall\overline{x_i}.\exists\overline{y_j}.\\
    \text{Context} &\Sigma ::= \overline{\mathcal{D}}\\
    \text{Model} &\mathcal{M} ::= \langle h, s \rangle\\
    \text{Stack} &s : \text{PV} \rightharpoonup \text{Val}\\
    \text{Heap} &h : \text{Loc} \rightharpoonup \text{Val}\\
    \text{Synthesis goal} &\mathcal{G} ::= P \leadsto Q
  \end{array}
\]

\subsection{Equivalence Relation}
\label{sec:SuSLikEquiv}

\begin{definition}[Synthesizability]
  We say a goal $\mathcal{G}$ is \textnormal{synthesizable} iff $\exists c. (\vdash \mathcal{G}\synth c)$
\end{definition}

\begin{definition}[SuSLik Command Equivalence]
  Given a context of inductive predicates $\Sigma$ and SuSLik commands $c$ and $c'$, $\Sigma \vdash c \approx_\mathcal{C} c'$ iff
    $\forall \mathcal{M}. \exists \mathcal{M'}.
      ((c, \mathcal{M}) \BigStep \mathcal{M'})
      \land
      ((c', \mathcal{M}) \BigStep \mathcal{M'})$
\end{definition}

\begin{definition}[SuSLik Goal Equivalence]
  Given a context of inductive predicates $\Sigma$ and synthesizable SuSLik goals $\mathcal{G}$ and $\mathcal{G}'$, $\Sigma \vdash \mathcal{G} \approx_{S} \mathcal{G}'$ iff
    $(\forall c\; c'. ((\Sigma \vdash \mathcal{G} \synth c) \land (\Sigma \vdash \mathcal{G}' \synth c')) \implies (\Sigma \vdash c \approx_{\mathcal{C}} c')$
\end{definition}

\section{Translation of fun-SuSLik into SuSLik}
\label{sec:Translation}

We give a denotational semantics for fun-SuSLik, where the denotations are SuSLik goals, with
the denotation function $\sem{\blank} : \ttt{e} \ra (\Sigma \times \mathcal{G})$.
This translation is shown to preserve equivalence in two steps (Section~\ref{sec:TransSoundness}):

\begin{itemize}
  \item Equivalent fun-SuSLik expressions are mapped to equivalent SuSLik goals. This amounts to proving
    \textit{full abstraction} for $\sem{\blank}$.
  \item Non-equivalent fun-SuSLik expressions are mapped to non-equivalent SuSLik goals.
    This is accomplished by proving $\sem{\blank}$ is \textit{adequate}.
\end{itemize}

% \subsection{Denotation by Inductive Predicates}
%
% Some fun-SuSLik constructs will translate to inductive predicates. This is translation is
% given by $\Isem{\blank} : \ttt{e} \ra (\ttt{I} \times \ttt{Name})$. This function produces both
% the inductive predicate and its name. Its name is determined by the expression. This
% name is unique as the inductive predicate is lambda lifted; that is, all captured variables
% are turned into arguments to the inductive predicate.

% \subsection{Denotation With Environments}
%
% A denotation function $\Esem{\cdot} : \ttt{e} \ra \mathbb{N} \ra \ttt{SP} \times \mathbb{N}$ is defined using
% elements of $\mathbb{N}$ as a source of fresh variable names. We take $\fresh : \mathbb{N} \ra \Name \times \mathbb{N}$ as a function
% which will generate a fresh variable name together with an updated environment.
%
% We use the convention that the result value of $\Esem{e}(n)$ will be stored in $n$.
%
% % For convenience, we define the following operator for applying a function in an environment to a value in an environment:
% %
% % \begin{definition}[Environment Bind]
% %   $\ttt{envApply} : (\ttt{e} \ra $
% % \end{definition}
%
% \begin{definition}[Denotation Function for fun-SuSLik]
%   $\sem{e} = proj_1(\Esem{e}(0))$
% \end{definition}
%
% \noindent
% We define $\Esem{\cdot}$ as follows.
%


\subsection{Soundness}
\label{sec:TransSoundness}

% \begin{definition}[$\alpha$-Equivalence of Contexts]
%   If $\Sigma$ and $\Sigma'$ are equal up to renaming, then we write $\Sigma \approx_{\alpha} \Sigma'$
% \end{definition}

% \begin{theorem}[Context Preservation]
%   $e \approx_F e' \implies \sem{e} = \sem{e'}$
% \end{theorem}

\begin{theorem}[Full Abstraction]
  If $e$ and $e'$ are fun-SuSLik expressions and $\Sigma \vdash e \approx_F e'$,
  then $\sem{\Sigma} \vdash \sem{e} \approx_{S} \sem{e'}$
\end{theorem}

\begin{theorem}[Adequacy]
  If $e$ and $e'$ are fun-SuSLik expressions and\\ $\Sigma \vdash e \not\approx_F e'$,
  then $\sem{\Sigma} \vdash \sem{e} \not\approx_{S} \sem{e'}$
\end{theorem}

\subsection{Translation}

\begin{math}
  \Csem{\ttt{data T := }\overline{\ttt{alts}_i}\ttt{;}} = p^{\alpha}(x, \Nsem{0}{\,\overline{\ttt{alts}_i}\,}) : \overline{\Asem{i}{\ttt{alts}_i}}
  \\
  \\
  \Nsem{k}{\ttt{C } \overline{\ttt{fields}_i}\; |\; \overline{\ttt{alts}_j}} = v_{k+1}, v_{k+2}, \dots, v_{k+i}, \Nsem{k+i}{\overline{\ttt{alts}_j}}
  \\
  \Nsem{k}{\cdot} = \cdot
  \\
  \\
  \Asem{i}{\ttt{C }\overline{\ttt{fields}_j}\,} = \ttt{(which == i)} \Ra \ttt{[x, j+1]} * x \mapsto \ttt{which} * \overline{\Fsem{i,j+1}{\ttt{fields}_j}}
  \\
  \\
  \Fsem{i,j}{\Int} = (x+j) \mapsto v_{i+j}
  \\
  \Fsem{i,j}{\Bool} = (x+j) \mapsto v_{i+j}
  \\
  \Fsem{i,j}{\ttt{T } \overline{\ttt{args}_k}} = (x+i) \mapsto v_{i+j} * \ttt{T}^{\alpha}(v_{i+j}, \Tsem{r}{\overline{\ttt{args}_k}})\qquad \text{with $r$ fresh}
  \\
  \\
  \Tsem{r}{f\; \overline{z}} = f, \Tsem{r}{\,\overline{z}\,}\qquad \text{where $f$ is a variable name}
  \\
  \Tsem{r}{\ttt{T } \overline{\ttt{args}_i}} =
\end{math}
\\
\textbf{TODO: Deal with nested type applications in the last equation above}

% Two cases, \nil\; and \cons:
%
% \begin{itemize}
%   \item \nil
%     \\
%     \[
%       \sem{\ttt{foldr add 0 nil}} = 
%     \]
% \end{itemize}

\end{document}


